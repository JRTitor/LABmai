documentclass{article}
\usepackage[utf8]{inputenc}
\usepackage[T1,T2A]{fontenc}
\usepackage[english,russian]{babel}
\usepackage[top=3.5 cm,bottom=2.5 cm,left=4 cm,right=4 cm,columnsep=25pt]{geome
try}
\usepackage{amsthm}
\usepackage{amssymb}
\usepackage{fancyhdr}
\newtheorem{theorem}{ТЕОРЕМА}
\begin{document}
\setcounter{page}{256} 
\pagestyle{fancy}
\fancyhead{}
\fancyhead[LO]{\thepage}
\fancyhead[CO]{ЧАСТНЫЕ ВИДЫ ДИФФЕРЕНЦИАЛЬНЫХ УРАВНЕНИЙ [ГЛ. VI]}
\setlength{\headheight}{13pt}
\fancyfoot{}
Если $a^2$ > $\frac{1}{4}$, то корни $k_1$ и $k_2$ $-$ комплексные; решение
$$y_1 = x^\frac{1}{2} + \cos({\sqrt{ {a^2} - \frac{1}{4} } \ln x}) ; \qqua
d y_2 = x^\frac{1}{2} + \sin ( { \sqrt { {a^2} - \frac{1}{4} } \ln x}) $$
имеют бесчисленное множество нулей в интервале $(1, \infty)$
Если ${a^2} < \frac{1}{4}$ , мы имеем решения н е к о л е б л ю щ и е с я:
$$y_1 = {x^{ \frac{1}{2}+\sqrt{\frac{1}{4}-{a^2}}}} ; \qquad y_1 = {x^{ \frac{
1}{2}-\sqrt{\frac{1}{4}-{a^2}}}} $$
точно так же, если $a^2 = \frac{1}{4}$, то
$$y_1 = x ^\frac{1}{2} , \qquad y_2 = x^\frac{1}{4} \ln{x}$$
Сравнивая с уравнением (53) уравнения вида (40), можем сказать: \emph{если, \ne
wline
начиная с некоторого $x$, постоянно имеем ${0} < {Q(x)} \leqslant {\frac{1}{4x^
2}} $ , то решение\newline
уравнения (40) не может иметь бесконечного числа нулей; если, начиная \newline
с некоторого значения $x$, имеем $Q(x) > \frac{1+x}{4x^2}$, где $a > 0$, то реш
ение \newline
уравнения (40) имеет бесчисленное множество нулей (теорема Кнезиса), }
Так уравнение $y^n + \frac{A}{x^a} y = 0$ не может иметь решений с бесконеч
ным \newline
числом нулей на интервале $(1, \infty)$.
Заметим, что все приведённые условия являются только достаточными \newline
для существования колеблющихся или неколеблющихся решений; они не дают \newl
ine
ответа на вопрос о колебаниях, если функция $Q(x)$ меняет знак или если её \
newline
нижняя граница на интервале $(a, \infty)$ равна нулю, а верхняя положительна
я.
$4. $Т е о р е м а Ш п е т а Если в уравнении (40) $Q(x) \equiv 0 $ , то
оно \newline
имеет фундаментальную систему решений $1, x;$ теорема Шпета утверждает,\newli
ne
что если $Q(x)$ достаточно быстро стремится к $0$ при $x \to \infty$, то, нез
ависимо\newline
от знака $Q$, фундаментальная система соответствующего уравнения при \newline
больших значениях $x$ мало отличается от $1, x.$ Введем обозначение: если \ne
wline
при $x \to \infty$ отношение $\frac{f(x)}{x^2}$ остаётся ограниченным, мы буд
ем писать \newline
$f(x) = O(x^2)$
\newtheorem*{myth}{Т Е О Р Е М А}
\begin{myth} Если $Q(x) = O(\frac{1}{x^{k+2}}) $, где $k > 0$ $(0 \leqslant{x}
< \infty)$, то урав-\newline
нение (40) обладает такой фундаментальной системой $y_1, y_2$, что \newline 
$y_1(x) - 1 = O (\frac{1}{x^k})$; \qquad $y_2(x) - x = O (\frac{1}{x^{k-1}
})$ при $k \neq 1$, и $y_2(x) - x = $\newline
$= O(\ln{x})$ при $k = 1$ \end{myth}
Для доказательства рассмотрим более общее уравнение, содержащее \newline
параметр $\lambda$,
$$ y^n = \lambda Q(x)y, \eqno(54)$$
которое обращается в заданное при $\lambda = - 1 $. Пусть $Q(x)$ опреде
ленно для
$ 0 \leqslant x < \infty$. Ищем уравнение (54) в виде ряда по степеням
$\lambda$
$$ y = y^{(0)} + \lambda y ^{(1)} + \dots + \lambda^{n} y^{(n)} + \dot
s \eqno(55) $$
\newpage
\setcounter{page}{257}
\pagestyle{fancy}
\fancyhead{}
\fancyhead[RO,LO]{\thepage}
\fancyhead[LO]{\S 21 \ ЛИНЕЙНЫЕ УРАВНЕНИЯ ВТОРОГО ПОРЯДКА}
\fancyfoot{}
Сначала построим $y_1;$ положим ${y_1}^{(0)} = 1;$ подставляя выражение (55) в
урав- \newline
нение (54) и приравнивая коэффициенты при одинаковых степенях $\lambda$, получи
м \newline
рекурентные уравнения для определения ${y_1}^{(0)} ;$
$${y_1}^{{(1)}^n} = Q(x), \ {y_1}^{{n}^n} = Q(x)y^{(n-1)}, \ n = 2, 3, \dots ;
$$
$$ y_1 = 1 + \lambda {y_1} ^{(1)} + \dots + \lambda^{n} {y_1}^{(n)} + \dots \eq
no(55_1) $$
Функции ${y_1}^{(n)}$ последовательно найдутся квадратурами, которые мы \newlin
e
в виде:
$${y_1}^{(1)} = {\int\limits_x^{\infty} \,d \xi } {\int\limits_\xi^\infty Q(t)\
,dt; } \ {y_1} ^{(n)} = {\int\limits_x^{\infty} d \xi } {\int\limits_\xi^\inft
y Q\,(t){y_1}^{(n-1)}(t)\, dt, }\ n = 2, 3 \dots \eqno(56) $$
Докажем, что несобственные интегралы в формулах $(56)$ сходятся, \newline
и оценим $|{y_1}^{(n)}|$ .
Из условия $Q(x) = O (\frac{1}{x^{k+2}})$ следует существование такой положи-\n
ewline
тельной постоянной $A$, что при $0 \leqslant x < \infty $ имеет неравенство:
$$|Q(x)| < {\frac{A}{{(1+x)}^{k+2}}} . $$
Отсюда находим :
$$|{y_1}^{(1)}| < {\int\limits_x^{\infty} d \xi }{\int\limits_\xi^\infty} {\fra
c{Adt}{(1+t)^{k+2}} = A {\int\limits_x^{\infty} \frac{d \xi}{(k+1)(1+\xi)^{k+1}
}}} = \frac{A}{k(k+1)} \frac{1}{{(1+k)}^{k}};$$
$$|{y_1}^{(1)}| < {\int\limits_x^{\infty}d\xi } {\int\limits_\xi^\infty} {\frac
{A}{(1+t)^{k+2}}{\frac{A}{k(k+1)} \frac{1}{{(1+k)}^{k}}} dt} = \frac{A^2}{k(k+1
)2k(2k+1))} \frac{1}{{(1+k)}^{2k}} .$$
Полной индукцией легко доказывается оценка
$$ |{y_1}^{(n)}| < \frac{A^n}{k(k+1)2k(2k+1))\dots nk(nk+1)} \frac{1}{{(1+k)}^
{nk}} .$$
Из этих оценок следует, что ряд $(55_1)$ сходится абсолютно и равномерно \newli
ne
для $0 \leqslant x < \infty $ при любом $\lambda$ (и в частности при $\lambda =
-1 $ и представляет ре-\newline
шение уравнения $(54)$
Наконец, из неравенства
$$ |y_1 - 1| < \frac{|\lambda|A}{k(k+1)}\frac{1}{(1+x)^k}
\left \{ 1 + \frac{|\lambda|A}{2k(2k+1)}\frac{1}{(1+x)^k} + \right .$$
$$\left . + \frac{|\lambda|^2 A^2}{2k(2k+1)3k(3k+1)}\frac{1}{(1+x)^{2k}} + \dot
s \right \},$$
где в фигурных скобках стоит сходящийся ряд, сумма которого стремтся \newline
к $1$ при $x \to \infty $, получаем
$$ y_1(x) - 1 = O\left (\frac{1}{x^k}\right ) . $$
Для решения $ y_1(x)$ теорема доказана.
\end{document} 
